% Options for packages loaded elsewhere
\PassOptionsToPackage{unicode}{hyperref}
\PassOptionsToPackage{hyphens}{url}
%
\documentclass[
]{article}
\usepackage{amsmath,amssymb}
\usepackage{lmodern}
\usepackage{ifxetex,ifluatex}
\ifnum 0\ifxetex 1\fi\ifluatex 1\fi=0 % if pdftex
  \usepackage[T1]{fontenc}
  \usepackage[utf8]{inputenc}
  \usepackage{textcomp} % provide euro and other symbols
\else % if luatex or xetex
  \usepackage{unicode-math}
  \defaultfontfeatures{Scale=MatchLowercase}
  \defaultfontfeatures[\rmfamily]{Ligatures=TeX,Scale=1}
\fi
% Use upquote if available, for straight quotes in verbatim environments
\IfFileExists{upquote.sty}{\usepackage{upquote}}{}
\IfFileExists{microtype.sty}{% use microtype if available
  \usepackage[]{microtype}
  \UseMicrotypeSet[protrusion]{basicmath} % disable protrusion for tt fonts
}{}
\makeatletter
\@ifundefined{KOMAClassName}{% if non-KOMA class
  \IfFileExists{parskip.sty}{%
    \usepackage{parskip}
  }{% else
    \setlength{\parindent}{0pt}
    \setlength{\parskip}{6pt plus 2pt minus 1pt}}
}{% if KOMA class
  \KOMAoptions{parskip=half}}
\makeatother
\usepackage{xcolor}
\IfFileExists{xurl.sty}{\usepackage{xurl}}{} % add URL line breaks if available
\IfFileExists{bookmark.sty}{\usepackage{bookmark}}{\usepackage{hyperref}}
\hypersetup{
  pdftitle={Development of a data-container to efficiently manage subsets of scRNA-Seq data},
  pdfauthor={Irzam Sarfraz},
  hidelinks,
  pdfcreator={LaTeX via pandoc}}
\urlstyle{same} % disable monospaced font for URLs
\usepackage[margin=1in]{geometry}
\usepackage{graphicx}
\makeatletter
\def\maxwidth{\ifdim\Gin@nat@width>\linewidth\linewidth\else\Gin@nat@width\fi}
\def\maxheight{\ifdim\Gin@nat@height>\textheight\textheight\else\Gin@nat@height\fi}
\makeatother
% Scale images if necessary, so that they will not overflow the page
% margins by default, and it is still possible to overwrite the defaults
% using explicit options in \includegraphics[width, height, ...]{}
\setkeys{Gin}{width=\maxwidth,height=\maxheight,keepaspectratio}
% Set default figure placement to htbp
\makeatletter
\def\fps@figure{htbp}
\makeatother
\setlength{\emergencystretch}{3em} % prevent overfull lines
\providecommand{\tightlist}{%
  \setlength{\itemsep}{0pt}\setlength{\parskip}{0pt}}
\setcounter{secnumdepth}{-\maxdimen} % remove section numbering
\ifluatex
  \usepackage{selnolig}  % disable illegal ligatures
\fi

\title{Development of a data-container to efficiently manage subsets of
scRNA-Seq data}
\usepackage{etoolbox}
\makeatletter
\providecommand{\subtitle}[1]{% add subtitle to \maketitle
  \apptocmd{\@title}{\par {\large #1 \par}}{}{}
}
\makeatother
\subtitle{v1: Basic pre-processing and heatmaps}
\author{Irzam Sarfraz}
\date{May 16, 2021}

\begin{document}
\maketitle

{
\setcounter{tocdepth}{2}
\tableofcontents
}
\hypertarget{abstract}{%
\section*{Abstract}\label{abstract}}
\addcontentsline{toc}{section}{Abstract}

\hypertarget{introduction}{%
\section{Introduction}\label{introduction}}

Over the past decade or so, due to the extensive availability of
computational resources, computational approaches have contributed
significantly to offer efficient and effective solutions to intricate
research questions in domains that do not directly offshoot from
traditional computer science paradigm. Computational Biology,
Computational Biomedicine and Bioinformatics are some of the recently
developed domains that have emerged directly from the amalgamation of
Biology, Computer Science and Statistics due to a strenuous need of
solving complex biological problems in the era of big data and
informatics.

Some of the grand challenges that been worked on extensively over the
past few decades include genomic sequence analysis, homology search and
protein structure prediction. Overall, end goal of all of these
challenges is to infer a better understanding of the underlying
biological complexity and to somehow use this understanding to combat
diseases that were previously uncurable or untreatable.

At the same time, the availability and the use of computational
resources have increased by manifold over the last decade and therefore
this provides an excellent opportunity to use these resources to solve
the above-mentioned complex problems. Genomic sequence analysis is one
the problems that is of particular interest to the researchers around
the world due to its explanatory power of various diseases, their
prognosis and outcomes.

Genomic sequence extraction technologies have improved drastically since
the beginning of this millennium and have only continued to grow. These
technologies offer extraction of genetic data at very high resolution
(even possible at cellular level) and the analysis of this data can
result in extremely useful biological insights. For example, Precision
Medicine is an emerging domain that combines genetic data extracted
through these technologies with the patient clinical data to predict
patient outcome.

These computational analyses that lead to useful biological insights are
often quite complex, use large amounts of data and consume vast
computational resources. It is therefore, utmost important, to tackle
these analyses by developing methods, tools and data structures that can
conserve one or more of these resources. Additionally, because most of
these analyses are quite complex, lengthy and require the use of
supplementary data (in addition to the primary genomics data), it
becomes quite tricky to keep track of the different shapes of the data
that have emerged over the analysis pipeline. Therefore, in this study,
we focus on developing a data-structure that efficiently manages subsets
of high-throughput genomic sequence data by eliminating the need to
redundantly store additional data which is a common occurrence in many
of the computational analyses of biological high-throughput data and
somehow deliver features for provenance tracking for easier tracking
different shapes of the data throughout an analysis workflow.

\hypertarget{basic-concepts-and-terminologies}{%
\subsection{Basic concepts and
terminologies}\label{basic-concepts-and-terminologies}}

As the proposed study/project has applications in Bioinformatics, a
brief introduction and explanation of the key concepts and terminologies
is required. Some of these are given below.

\hypertarget{high-throughput-genomic-data}{%
\subsubsection{High-throughput genomic
data}\label{high-throughput-genomic-data}}

High-throughput genomic technologies such as microarrays and sequencing
allow rapid generation of vast amounts of data from both DNA and RNA of
an origanism that is both large in size and high-dimensional. These
high-throughput genome extraction technologies are widely used to
profile the genome of an organism, to better understand the genetic
structure of an organism or even to identify and address the causes of
certain diseases and defects. In contrast, the traditional PCR-based
approaches are more concerned with amplification and detection of a few
transcripts of interest as compared to hundreds of thousands in
high-throughput approaches. Most of the times, the output of
high-throughput sequeuncing experiments are counts matrices where each
column of the matrix represents a particular sample and the row
represents the quantified presence of a particular gene or transcript in
that sample. These matrices form the basis of a large amounts of
statistical analyses that can be performed to gain insight about the
underlying biology of the samples.

\hypertarget{assay}{%
\subsubsection{Assay}\label{assay}}

An assay (or a data assay) is the actual form of the data that can be
used with a particular statistical analysis method. Initially, this
assay is the counts matrix which is output from a sequencing experiment
but can undergo several transformations throughout its lifetime during
an analysis workflow. For example, the initial assays (counts matrix)
can undergo normalization, log-transformation, scaling, trimming among
others and are stored as separate assays which be called and used during
the analysis workflow as per requirements of the workflow and the method
being used. However, it must be understood that these multiple assays
belong to the same experiment but only represent the multiple
transformations that have been performed on the original assay so as to
have original assay in all forms. The assays always have dimensions of
m*n where m is the number of genes (rows) and n is the number of samples
(columns).

\hypertarget{coldata}{%
\subsubsection{colData}\label{coldata}}

The `colData' is short for column data and is always the metadata
associated with the samples (columns) of the dataset (columns of an
assay). The `colData' has dimensions n*q where n is the number of
samples and q \textgreater= 0 based on the number of metadata columns
(for samples) available for a particular dataset. The `colData' is often
referred to as the phenotype data as it is associated with the
phenotypic characteristics (patient data) of the dataset. For example,
the `colData' may have the disease condition of each sample in the
dataset (tumor or normal) or age of the patient that the sample belongs
to. This `colData' may be used in many downstream analysis methods or
visualizations for labeling of samples or even in statistical methods to
separate the samples based on a phenotype, e.g., a linear model may be
used between normal and tumor tissues (referenced from colData) to
identify significant genes based on the quantified assay data.

\hypertarget{rowdata}{%
\subsubsection{rowData}\label{rowdata}}

The `rowData' as obvious is the metadata associated with the genes
(rows) of the dataset (rows of the an assay). The `rowData' has
dimensions m*q where m is the number of genes and q\textgreater=0 abased
on the number of metadata columns (for genes) available for a particular
dataset. The `rowData' is often used to store the computed statistics
for the genes in the dataset. For example, during the highly variable
genes computation, the variance and the mean computed for each gene is
stored in separate columns of the `rowData' which may be used at a later
stage to identify the most highly variable genes from these statistics.
Additionally, `rowData' may be used to store anything that may be
related to the genes of the dataset, e.g., genes are often referenced by
several different systems of naming conventions (ENSEMBLE, Entrez or
Gene Symbols) which can be stored in the `rowData'.

\hypertarget{bioconductor-cran}{%
\subsubsection{Bioconductor \& CRAN}\label{bioconductor-cran}}

Packages or libraries built for R environment are available for download
and use from multiple sources including R CRAN, Bioconductor or even
directly from Github. The Comprehensive R Archive Network (CRAN) is the
largest repository for R packages maintained by R developers, while
Bioconductor is a community maintained (R core) repository (as well as a
database for sample datasets) and specializes in packages/libraries with
applications in biology. Submission to both of these repositories
undergo strenuous review (including novelty of the package, code quality
and testing) before it is accepted and becomes available for users to
download and use. Packages from both of these resources once available
can easily be downloaded from within R console environment.

\hypertarget{bioconductor-experiment-objects}{%
\subsubsection{Bioconductor Experiment
objects}\label{bioconductor-experiment-objects}}

As previously explained that Biocodunctor is a database/repository for R
packages and datasets having applications in biology, data-containers or
data-structures are required to particularly handle the such
high-dimensional datasets which can manage multiple data assays and the
associated phenotype data (colData) and the gene metadata. While, this
can be managed by using native R data-structures like matrix and
dataframes, it is much more intuitive to handle this multi-data datasets
using Experiment classes that are designed specifically for this
purpose.

These Experiment classes including SummarizedExperiment and
SingleCellExperiment are designed in such a way as to allow the users to
use the objects of these classes in a more biologically and
statistically relevant manner especially considering the workflow of
such datasets. For example, these classes allow storage of multiple
assays, colData, rowData as well as other types of data such as
dimensionality reduction results within the same object as all of these
data types are used within an analysis workflow from the same
experiment.

Moreover, because these data-containers are specialized, they acquire
multiple features necessary for such high-dimensional data such as use
of dgCMatrix (compressed sparse matrices) for assay data, dataframes for
colData/rowData and SimpleList for dimensionality reduction results, all
of which have additional advantages instead of using a single base
data-stucture for all types of data.

\hypertarget{background}{%
\subsection{Background}\label{background}}

High-throughput experiments like microarrays and RNA sequencing allow
simultaneous extraction of data from thousands of genes at the same
time, thus enabling researchers and scientists to study the effects of
mutations (change in gene structure) and expression changes (quantifying
the presence or absence of a gene) in genes that result in phenotype
variations {[}1{]}. These changes often help us in understanding the
underlying causes of numerous diseases, particularly heterogeneous
diseases such as cancer, where different cells may represent individual
changes at the genetic level {[}2{]}.

Single Cell RNA Sequencing (scRNA-Seq) is effectively a recent
advancement that allows data extraction at a very high resolution
i.e.~at the cellular level and reveals expression changes in genes
against each individual cell in the sample space. In contrast,
microarrays {[}3{]}, and bulk RNA sequencing {[}1{]} both average the
expression values extracted from the tissue of interest. This
high-resolution capture of genomic data at cellular level (scRNA-Seq)
has enabled researchers belonging to a broad range of scientific domains
to come together and collaborate across the board to solve biological
problems including but not limited to drug discovery {[}4{]}, artificial
intelligence in precision medicine {[}5{]}, targeted patient therapies
and biological data mining to better understand the disease progression
and outcomes {[}6{]}.

All of the previously mentioned biological problems require a
computational analysis of some sort to reach an insightful conclusion
particularly owing to the fact that these analyses often run on
scRNA-Seq datasets that are generally of the order of hundreds if not
thousands of gigabytes {[}7{]}. These computational analyses can range
from a simple visualization of data for quality control analysis or
differential expression to identify interesting genes and not to mention
even for the understanding of cell trajectories over pseudotime {[}8{]},
which is a rather complex task than the former two. To design and run
these analyses, several computer scripting languages such as R {[}9{]}
and Python {[}10{]} support manipulation of scRNA-Seq data due to the
extensive availability of relevant packages, libraries, and toolkits.
Bioconductor {[}11{]} platform serves as a central repository for all
such tools and packages to analyze high-throughput genomic data in
general for scripting in R language.

The first and the foremost task of any such analysis is the choice of a
data structure or more appropriately, a container to hold the scRNA-Seq
data for manipulation during all subsequent tasks of an analysis. This
is relatively a tricky decision because most of the toolkits and
analytical pipelines for analysis of scRNA-Seq data support only a few
of the proposed containers and occasionally provide their internal
objects to allow easier manipulation within the toolkit.

However, issues arise when a range of analytical methods are integrated
in a single toolkit or an analytical pipeline, as is the case with many
of the proposed command line toolkits or Shiny web-interfaces {[}12{]}.
One issue that is of particular interest, is the generation and
management of subsets of scRNA-Seq data, the manipulation of which is
abundant in many of the common tasks of analysis pipeline such as during
the identification of highly variable genes {[}13{]} or the integration
of multiple assays {[}14{]} or even basic filtering steps. In such
cases, the available choice of containers does not provide flexibility
in terms of storage of such subsets and the consequent usage in the
downstream analysis. In many cases where there are several libraries and
methods integrated together to form a streamlined analysis pipeline, as
in the case of many toolkits and Shiny applications {[}15{]}, multiple
objects from these containers at times hold the same data albeit in
subset form. This has widespread consequences in terms of increased code
complexity against the underlying analysis, data redundancy, and
increased memory consumption as well as complexity that arises from the
handling of multiple objects created due to the subsets of the data.
Furthermore, because of the creation of multiple objects that
essentially point to the same data, provenance tracking is often
overlooked.

Typically, in such analyses, the utmost concern of the researcher is to
apply the appropriate statistical technique to gain useful biological
insights. Alternatively, in the case of a toolkit programmer, the main
concern is the addition of functionality to support such analyses. In
both scenarios, data handling and management is ultimately at the bottom
of the workflow stack. Therefore, the complexity and redundancy that
emerges from the creation of a collection of objects due to the nested
nature of the subsets that are essentially pointing to the same data,
results in disastrous implications as far as time and memory consumption
are concerned. These factors become especially significant when datasets
are enormous {[}7{]} and the corresponding analyses are tricky, as is
the case with scRNA-Seq datasets.

\hypertarget{problem-statement}{%
\subsection{Problem statement}\label{problem-statement}}

The available data-structure classes for manipulation of high-throughput
sequencing data do not support storage, management and provenance
tracking of subsets of the data which are quite common in a typical
analysis workflow. Therefore, there is a need to develop a
data-structure class that can efficiently manage subsets of such data
while maintaining data provenance.

\hypertarget{research-objectives}{%
\subsection{Research objectives}\label{research-objectives}}

\begin{itemize}
\tightlist
\item
  Development of a data container that supports storage of subsets of
  sequencing data and allows efficient manipulation of subsets while
  significantly reducing the memory footprint.
\item
  Interface to the user must be same as offered by SingleCellExperiment
  and SummarizedExperiment classes to ensure easy replacement of
  existing code with the new package.
\item
  Ensure conservation of memory by eliminating the need to store
  redundant data values against all the data assays, rowData and
  colData, by using pointers to the existing data whenever and wherever
  possible.
\item
  Provenance tracking to ensure that the origin of the data can easily
  be tracked.
\item
  Apply this data container to commonly used analysis workflows and show
  how efficiently it manages subsets of data from within a single object
  while saving memory.
\end{itemize}

\hypertarget{research-question}{%
\subsection{Research question}\label{research-question}}

How can we develop a data-structure that can efficiently manage subsets
of high-throughput genomic data while preserving data provenance and
ensuring no additional redundant data is stored?

\hypertarget{scope-of-research}{%
\subsection{Scope of research}\label{scope-of-research}}

\begin{enumerate}
\def\labelenumi{\arabic{enumi}.}
\tightlist
\item
  The proposed data-container shall be available as an R package through
  the Bioconductor repository accompanying a vignette that illustrates
  the description and common usage of the available methods with
  examples.
\item
  Inheritance from the commonly used data-containers shall ensure that
  interface is offered to the users for manipulation is same as the
  input class.
\item
  Single Cell RNA Sequencing data is the primary data supported by the
  package, yet all Bioconductor experiment datasets should work fine if
  they follow the experiment design of the SingleCellExperiment and
  SummarizedExperiment classes i.e., numeric matrices for assay data,
  and numeric/character data frames for rowData and colData.
\item
  Package shall be platform independent and should work with all
  operating systems supported by R language.
\end{enumerate}

\hypertarget{application-of-proposed-research}{%
\subsection{Application of proposed
research}\label{application-of-proposed-research}}

As the direct beneficiaries of our research, the analysts, researchers
and developers, can indeed build analysis pipelines and workflows for
genetic sequencing data in a much better, convenient and efficient
manner by using our approach as a drop-in replacement and a building
block for already available containers for such data that often include
subsets of such data. Indirectly, it is the patients and the general
public that is the recipient of the outcomes of the research of this
advancement and consequent improvements in such approaches, including
ours that impacts the disease prognosis, its outcome and its treatment.
While the abstract utility of our proposal lies in Bioinformatics
(particularly in biological data mining and analysis), the proposed
implementation and the direct application indeed acquires knowledge from
a range of multidisciplinary domains (data mining, computational biology
and statistics) all of which have a common root in Computer Science.

\hypertarget{thesis-outline}{%
\subsection{Thesis outline}\label{thesis-outline}}

\begin{itemize}
\tightlist
\item
  Chapter 1: Deals with the introduction, basic terminologies,
  background knowledge, problem statement, research objectives, research
  question and scope of the research (given in the Section
  1.1,1.2,1.3,1.3,1.5, and 1.6 respectively).
\item
  Chapter 2: Discusses the literature review according to the
  statistical, machine learning, and graph-based approaches (given in
  the Section 2.1, 2.2 and2.3, respectively).
\item
  Chapter 3: Deals with the detailed description of proposed research
  methodology from data collection to implementation (given in the
  Section 3.1 to 3.8, respectively).
\item
  Chapter 4: Deals with the representation and discussion of results
  (given in the Section 4).
\item
  Chapter 5: Discusses the conclusion, limitation, and future work
  (given is the Section 5)
\end{itemize}

\hypertarget{literature-review}{%
\section{Literature Review}\label{literature-review}}

In a typical genomic experiment, a matrix-type data-structure represents
the data extracted from the experiment, where the rows of this
matrix-type data-structure represent the genes or features and columns
represent the samples. The row-column intersection of such a matrix
records the data-value for that gene in the row against a sample. The
data-value can be either the expression value of a gene in the case of
microarray data or counts of a gene if the data is from a sequencing
experiment. Additionally, these values may also represent transformed
data, such as when the matrix undergoes log-transformation,
normalization, or scaling as per the requirements of the downstream
analysis.

SummarizedExperiment {[}16{]} originally proposed to serve as an
all-in-one container for sequencing data supports the storage and
manipulation of multiple data assays as well as rowData and colData.
SummarizedExperiment follows the ExpressionSet {[}17{]} class available
for microarray experiments, technically following a similar structure
but allows more flexible manipulation in terms of management of
additional assays and feature information within a single object. Built
on top of SummarizedExperiment (more accurately the
RangedSummarizedExperiment class) is the SingleCellExperiment {[}18{]}
class that in addition to the features offered by the former, allows the
storage of dimensionality reduction computations. Moreover,
SingleCellExperiment class offers storage of alternate experiments such
as in the case of spike-in transcripts {[}19{]} that have different
dimensions from the original assays.

The design of both SingleCellExperiment and SummarizedExperiment
mandates the use of the object from these classes as a container for a
single experiment that can have multiple data assays. For example,
PBMC3K {[}7{]} dataset has 13714 features (genes) and 2700 samples
(cells) available for use and the data values are represented by a
counts matrix, each cell of which contains the number of molecules
identified against a feature and a sample. This represents a single
experiment and all relevant data for this experiment i.e.~the actual
data values as represented by the counts matrix (assay), feature
metadata (rowData) and sample metadata (colData), group together in a
single SingleCellExperiment or SummarizedExperiment object. These
classes allow the storage of multiple assays i.e.~the transformed
versions of counts matrix such as the scaled counts matrix or log
transformed counts matrix, within a single object as they correspond to
the same experiment. In doing so, both classes impose restrictions on
the dimension of the objects and the data supported by these objects.
For example, in the case of the generation of a subset of a counts
matrix, as is the case of many analysis tasks such as during variable
genes identification, these classes do not allow the storage of the
subset counts matrix corresponding only to the variable genes back into
the original objects. Therefore, for this purpose, only a new object can
handle this subset data which again is not only the subset counts
matrix, but also the corresponding rowData and colData.

When an experiment constitutes multiple observations having varying
dimensions, MultiAssayExperiment {[}20{]} offers integration of these
observations through the sampleMap function available with the package.
More intuitively, integration of datasets within a single object as
offered by MultiAssayExperiment is more oriented towards selection of
features represented by a specific condition. For example, consider a
scenario where huge amounts of data are available against a particular
disease, yet it is extracted by different protocols thus resulting in
non-uniform datasets as far as features, possible sub-types of diseases
and the associated origin of the samples are concerned. Consider cancer
for example {[}21{]}, a heterogenous disease that results in different
progression and prognosis based upon different factors with geography
being one of them. In this case, MultiAssayExperiment offers easy
integration of this diverse data with respect to a defined condition,
such as the patient geography and allows usage of features only
represented by this condition.

In addition to the above packages that primarily serve as a
data-structure or a data-container for holding and manipulating genomic
data such as scRNA-Seq data, some specific pipelines and toolkits
developed for analysis of scRNA-Seq data also offer native objects for
the same purpose. A widely used toolkit for analysis of scRNA-Seq data
i.e.~Seurat {[}14{]}{[}22{]}, provides a native Seurat object for
manipulation of data within the toolkit. While it offers flexibility in
terms of storage of multiple assays without imposing restrictions on
assay dimensions, little support outside of the toolkit makes it a less
desirable option when compared with the dedicated containers such as the
widely used SingleCellExperiment container.

All the previously discussed packages and containers support the common
R paradigm for subsetting data, either with built-in functions or
through standard R syntax for subsetting matrices and data frames.
However, storing back these subsets into the original objects and then
using them for further data manipulation or transformation while keeping
the original data and the subset data does not fall into the design
considerations of these packages.

\hypertarget{summarizedexperiment}{%
\subsection{SummarizedExperiment}\label{summarizedexperiment}}

SummarizedExperiment is an S4-based R class to serve as a data-container
for sequencing experiments with properities similar to the ExpressionSet
class historically used for microarray experiments. SummarizedExperiment
offers a all-in-one solution to store and coordinate expression
measurements from sequencing experiments with support for many other
statistical functions and methods that can directly work on
SummarizedExperiment objects.

The slots available with SummarizedExperiment objects include assay slot
to hold multiple data assays (using SimpleList), colData slot to hold
column metadata (dataframe), rowData slot to hold row metadata
(datafrane) and metadata slot to hold general metadata (list) about the
dataset. Common methods provided by SummarizedExperiment class include
assay() method to get or set an assay, colData() method to get or set
colData, rowData() method to get or set rowData() and a metadata()
method to get or set metadata to SummarizedExperiment object.

However, SummarizedExperiment objects have a limitation of not letting
users to store assays of different sizes within a single object. This is
because of its uniform structure that coordinates the assays with
rowData/colData by keeping a object-level row/column size of the overall
object to avoid complications. As a result, when assays are subsetted as
in many cases during an analysis workflow, they cannot be stored back
into the original object.

\hypertarget{singlecellexperiment}{%
\subsection{SingleCellExperiment}\label{singlecellexperiment}}

SingleCellExperiment is a data-container specifically built for storage
and manipulation of single-cell data in contrast to the
SummarizedExperiment class which has applications throughout gene
expression measurements regardless of the data extraction protocol. As a
result, SingleCellExperiment has additional slots for single-cell data
manipulation while keeping all of the slots of SummarizedExperiment
intact by using a direct inheritance approach.

The additional slots that have been added to SingleCellExperiment class
are the `reducedDims' for storage and manipulation of dimensionality
reduction results and `altExps' for storage of additional alternative
experiments. Both of these slots have been added keeping in mind the
specific needs of single-cell data, which due to its extraction protocol
requires the addition of spike-in genes to the data assays to somewhat
transform the data in a shape better suited for downstream analysis by
adjusting for excess zeros which is a major characteristic of
single-cell data.

While the \texttt{altExps} slot allows the storage of a complete
experiment object of different dimensions, it is originally meant for
the manipulation of spike-in genes and therefore lacks coordination of
data between multiple dimensions. As a result, it does not support
direct manipulation of subsets of data similar to the
SummarizedExperiment class.

\hypertarget{multiassayexperiment}{%
\subsection{MultiAssayExperiment}\label{multiassayexperiment}}

MultiAssayExperiment is another Experiment class for storage of
expression data, but allows integration of data from multiple
experiments and provides a common interface to this integrated data to
the user. It is different from both SummarizedExperiment and
SingleCellExperiment classes in its overall objective, which is the
integration of datasets through a common variable instead of acting as a
basic container for day-to-day manipulation.

For example, if multiple datasets are available for lung cancer, they
can be integrated together through a sampleMap() function provided by
the class that uses an anchor column for phenotype data to merge and
integrate together these many datasets having different characteristics
such as age, type of lung cancer, site of tumor tissue and many others.

The MultiAssayExperiment allows the storage of multiple datasets having
different dimensions but it is more oriented towards integration of data
and therefore does not provide a straightforward approach towards
subsetting of single experiments.

\hypertarget{research-methodology}{%
\section{Research Methodology}\label{research-methodology}}

The overall research methodology for this study includes the package
design (high-level logical design to support subsets in existing
*Experiment packages), package implementation (development and
deployment of package in R environment, including package documentation)
and the validation of the efficiency of the package on different
datasets (results). The research methodology can be perceived easily
from the Figure.

\hypertarget{package-design}{%
\subsection{Package Design}\label{package-design}}

Keeping in view the objectives and scope of the package, an abstract
logical design of the package utility is presented in the figure.

In the figure above, an abstract design for the utlitiy of the package
is presented. The goal of the package is to provide the ability to
create subsets from available main data assays (counts, logCounts etc.
from the figure) and link them in a hierarchical order. This linking of
assays allows the creation of hierarchical subsets which are common
during large and complex analysis workflows. Essentially, subsets can be
created from either the main assays, or other subsets, where the the
newly created subset can have a subset of rows or columns or a
combination of both.

The figure above demonstrates the general class structure of the
package.

The figure above represents the class diagram of the new proposed class
to manage subsets of data.

\hypertarget{package-implementation}{%
\subsection{Package Implementation}\label{package-implementation}}

The R programming language supports multiple types of object-oriented
class systems including S3, S4, R5 and R6 (give comparison). For the
implementation of the said package, we use the S4 class system because
of the following two reasons:

\begin{enumerate}
\def\labelenumi{\arabic{enumi}.}
\tightlist
\item
  S4 class system is more OOP oriented than other classes, where you can
  have setter/getter methods with a single generic.
\item
  Experiment classes that we wish to inherit are also S4-based.
\end{enumerate}

Keeping in view the above arguments, we use the S4 class system to
implement our class by inheriting from other Experiment classes.

A brief diagram of implementing our package (and the class) is presented
in the figure below.

\hypertarget{setup-r-environment}{%
\subsubsection{Setup R environment}\label{setup-r-environment}}

\begin{enumerate}
\def\labelenumi{\arabic{enumi}.}
\tightlist
\item
  R environment for package build is setup using RStudio software and a
  github repository is linked for version control.
\item
  R supports multiple system of classes where each serves a specific
  purpose and needs. A comparison of these classes is provided in the
  table below.
\item
  As the requirements of our package is to serve as a drop-in
  replacement class for other Experiment classes, we use the same class
  system as these Experiment classes i.e., S4 class system that uses a
  systematic way of managing setters and getters and allows inclusion of
  slots.
\item
  The drop-in replacement feature would allow our package and classes to
  be used interchangeably with the existing methods that utilize other
  Experiment classes without having the need to convert between these.
\item
  As there are 5 Experiment classes namely SummarizedExperiment,
  RangedSummarizedExperiment, SingleCellExperiment,
  TreeSummarizedExperiment and SpatialExperiment, we inherit from each
  of these separately to build new subset classes that we name as
  SubsetSummarizedExperiment, SubsetRangedSummarizedExperiment,
  SusbetSingleCellExperiment, SubsetTreeSummarizedExperiment and
  SubsetSpatialExperiment. All five of these new Subset classes inherit
  directly from their parent Experiment classes so as to allow these
  subset classes to work exactly as intended by their original parent
  Experiment classes as a drop-in replacement with existing methods but
  adding subset support to these new Subset classes.
\end{enumerate}

\hypertarget{implement-constructor}{%
\subsubsection{Implement Constructor}\label{implement-constructor}}

\begin{enumerate}
\def\labelenumi{\arabic{enumi}.}
\tightlist
\item
  To enable easy conversion from other Experiment classes, we implement
  a simple constructor function ExperimentSubset() that takes input an
  object that belongs to any of the other Experiment classes and
  converts it into a Subset class appropriate to that object. This
  allows easy use of the object as a drop-in replacement for other
  Experiment classes in downstream analysis.
\item
  The constructor method also supports quick subsetting if the creation
  of a subset is required at the time of the creation of the object. For
  example, in case when other Experiment objects are being used in an
  analysis workflow and subset support is needed, users can immediately
  convert to a Subset class and create a specified subset from within
  the constructor method. For this purpose we have optional parameters
  in the constructor method that allow this creation of a subset.
\end{enumerate}

\hypertarget{internal-class-assaysubset}{%
\subsubsection{Internal class
AssaySubset}\label{internal-class-assaysubset}}

\begin{enumerate}
\def\labelenumi{\arabic{enumi}.}
\tightlist
\item
  To store subset information in our package we create another internal
  class called AssaySubset which serves not only for storing subset
  information but also as a conveninent way to link these subsets to the
  original parent data and other subsets.
\item
  The AssaySubset class is also an S4 class but it is not exported which
  means it cannot be directly used by the users. Instead, the
  constructor method of this class is called in our wrapper method
  createSubset() that internally uses this constructor method to
  initialize an object of class AssaySubset and stores the relevant
  subset information and characteristics in this AssaySubset object.
  This AssaySubset object that stores the subset information is saved
  inside the subset slot of Subset class which uses this slot and the
  information inside this slot to coordinate the subsets with the parent
  data.
\item
  Each AssaySubset object has further multiple slots that store the name
  of the subset, name of the parent assay, row indices from the parent
  assay that are part of this subset and column indices from the parent
  assay that are part of this subset. Additionally, each AssaySubset
  object stores a empty object of the original class to store
  non-redundant transformed data matrices within this subset. The
  row/column indices work similar to pointers and by using the parent
  assay name these indices are used to reference the redundant data from
  the original parent assay. This strategy eliminates the need to store
  any redundant data thus saves considerable amount of memory. The
  overall structure of this AssaySubset class and how the member slots
  of this class are used to reference data from parent assays is shown
  in the figure.
\end{enumerate}

\hypertarget{implement-subset-manipulation-methods}{%
\subsubsection{Implement subset manipulation
methods}\label{implement-subset-manipulation-methods}}

\begin{enumerate}
\def\labelenumi{\arabic{enumi}.}
\tightlist
\item
  createSubset() method is implemented for the creation of a subset. The
  input to this method is an ExperimentSubset object and specific
  parameters that specify the subset name, row/column indices of the
  parent assay and the name of the parent assay to which this subset
  belongs. The result of using this function is the creation of a new
  AssaySubset class object that stores the information of the new subset
  which is ultimately stored as an element in the input ExperimentSubset
  object.
\item
  getSubsetAssay() and setSubsetAssay() methods are implemented to get
  or set an assay specifically to a subset. S4 method approach is not
  used here since both getter and setter methods work differently and
  require different number of methods.
\item
  subsetColData() S4 method is implemented that allows setting and
  getting of colData specifically from a subset.
\item
  subsetRowData() S4 method is implemented that allows setting and
  getting of rowData specifically from a subset.
\item
  susbetColnames() S4 method is implemented that allows setting and
  getting of colnames specially from a subset.
\item
  susbetRownames() S4 method is implemented that allows setting and
  getting of rownames specially from a subset.
\item
  subsetAssayCount() method returns a total count of the subsets and the
  internal assays in these subsets.
\item
  subsetCount() method returns count of just the subsets from the input
  object.
\item
  subsetDim() method returns the dimensions of a particular subset.
\item
  subsetAssayNames() method returns the names of the subsets and the
  assays inside these subsets collectively.
\item
  subsetNames() method returns only the names of the subsets stored
  inside an input ExperimentSubset object.
\item
  subsetParent() method retrieves a complete subset to parent link of a
  specified subset. This method is particularly helpful in understanding
  the provenance of the data, i.e., how this particular subset
  originated and how it has been transformed over time.
\item
  subsetSummary() method is similary to the R base summary() method but
  it specifically provides a summary of an input ExperimentSubset object
  and the current state of the subsets. Specifically, it describes the
  size of the overall object, all subsets, their sizes, their internal
  assays and if any additional reducedDims have been stored in this
  subsets. Additionally, it shows the subset-parent hierahrical link of
  each subset retrieved from the subsetParent() method.
\item
  subsetSpatialData() method from SpatialExperiment class to get or set
  spatialData to specifically a subset.
\item
  subsetSpatialCoords() method from SpatialExperiment class to get or
  set spatialCoords to specifically a subset.
\item
  subsetRowLinks() method from TreeSummarizedExperiment class to get or
  set rowLinks to specifically a subset.
\item
  subsetColLinks() method from TreeSummarizedExperiment class to get or
  set colLinks to specifically a subset.
\end{enumerate}

\hypertarget{override-methods}{%
\subsubsection{Override methods}\label{override-methods}}

\begin{enumerate}
\def\labelenumi{\arabic{enumi}.}
\tightlist
\item
  To support subset capability with existing methods from all Experiment
  classes, we override these methods and add subset support to them made
  possible by use of direct inheritance from these classes.
\item
  For methods that exist with multiple classes, we create a single
  function and then call this function in the generic function of each
  of the class method to minimize the code redundancy and to increase
  code reusability.
\item
  The methods inherited from the Experiment classes to which subset
  support have been added through the strategy described above are
  specified below:
\end{enumerate}

\begin{enumerate}
\def\labelenumi{\alph{enumi}.}
\tightlist
\item
  show()
\item
  The show() is a base R (methods class) function that prints the object
  summary, visualizes a plot or prints other necessary information which
  is suitable for that particular class of objects. Here, we override
  the show() function and mimic the output of other Experiment classes
  but additionally add printing of subsets and subset assays for easier
  understanding of the current object structure.
\item
  assay()
\item
  An S4 method that can set or get an assay (data) depending upon on
  which side of the assignment operator (\textless-) it is used. In our
  package, we override this method to support setting or getting an
  assay from a subset aswell by using an additional parameter called
  `subsetName' in which you can specify the name of the subset to which
  or from which subset you want to set or extract an assay.
\item
  rowData()
\item
  An S4 method that can set or get rowData (row or genes or features
  metadata) associated with the current object. In this package, we
  additionally add support to set or get rowData specifically from a
  subset by specifying the name of the subset in the additional
  `subsetName' parameter. This data occasionally includes additional
  information abou the genes or features in the dataset and is also used
  to store computations limited to the genes for example, statistics
  computed for variable genes are stored in the rowData of the object.
\item
  colData()
\item
  An S4 method that can setor get colData (columns or cells or samples
  metadata) associated with the current object. In our package, we
  additionall add support to set or get colData specifically from a
  subset by specifying the name of the subset in the additional
  `subsetName' parameter. This data may also include additional
  information about the samples such as computed clusters or sample
  information.
\item
  metadata()
\item
  A general S4 method to store additional metadata about the dataset
  which cannot be included in the rowData or colData. This slot
  generally includes information about how the data was extracted by a
  particular protocol. Here, we allow the users to use the additional
  `subsetName' parameter to additionally store metadata for just a
  particular subset of interest.
\item
  reducedDim()
\item
  The reducedDim S4 method sets or gets the dimensionality reduction
  results to or from an Experiment object. Additional support for
  storage and retrieval of dimensionality reduction results for subsets
  is added through the `subsetName' parameter.
\item
  reducedDims()
\item
  The reducedDims S4 method sets or gets multiple reducedDims to or from
  an Experiment object. Additional support for doing so with subsets is
  added through the \texttt{subsetName} parameter.
\item
  reducedDimNames()
\item
  The reducedDimNames S4 method sets or gets the names of the stored
  dimensionality reduction results in an Experiment object. By using the
  `subsetName' parameter, names of the reducedDims can be retrieved or
  modified from a subset.
\item
  altExp()
\item
  The altExp S4 method sets or gets additional alternative experiment
  objects to the parent Experiment object. We have added support to add
  or remove (or retrieve) alternative experiments to each specific
  subset by using the `subsetName' parameter.
\item
  altExps()
\item
  The altExps S4 method sets or gets multiple alternative experiment
  objects to the parent Experiment object. We have added support for
  adding, removing or retrieving of multiple alternative experiments
  from a specific subset by using the `subsetName' parameter.
\item
  altExpNames()
\item
  The altExpNames S4 method gets or sets names of the stored alternative
  experiments in an Experiment object. Similarly, names of the stored
  alternative experiments can be set or get from subsets using the
  additional `subsetName' parameter.
\item
  spatialData()
\item
  The spatialData S4 method is explicitly available in the
  `SpatialExperiment' class to set or get the spatial data from an
  object. We have added subset support to set or get spatial data to or
  from a subset using the additional `subsetName' parameter as long as
  the object is inherited from the parent `SpatialExperiment' class.
\item
  spatialCoords()
\item
  The spatialCoords getter is a complimentary method to the spatialData
  method and retrives only the spatial coordinates from the overall
  spatial data. We have added subset support to this method which allows
  the users to retrieve the spatial coordinates just for a particular
  subset by specifying the name of the subset in the additional
  `subsetName' parameter.
\item
  rowLinks()
\item
  The rowLinks accessor method retrieves the information of the rows
  linked with the row tree in the `TreeSummarizedExperiment' class. Here
  we have added support for subsets using the `subsetName' parameter
  which lets us retrieve the rowLinks for just a particular subset as
  long as the object is inherited from the same parent class.
\item
  colLinks()
\item
  The colLinks accessor method retrieves the information of the columns
  linked with the col tree in the `TreeSummarizedExperiment' class. Here
  we have added support for subsets using the `subsetName' parameter
  which lets us retrieve the colLinks for just a particular subset as
  long as the object is inherited from the same parent class.
\end{enumerate}

\hypertarget{document-package}{%
\subsubsection{Document package}\label{document-package}}

R packages are generally documented in two standard ways, one through
`roxygen2' package that creates a documentation page for each function
or method and secondly through a package vignette which is more of a
user guide. Following both of these standards, we used the `roxygen2' R
package to create documentation (Rd files) for each of the
function/method in our package that describes the function/method name,
short description, input/output of the function/method and a runnable
example to show the working of the function/method. These `.Rd' files
generated by `roxygen2' then become available online repositories for
users to view such as the `rddr.io'.

Additionally, to document the usage of the package, we used Rmarkdown to
create a vignette to serve as a user manual. In the vignette, we
describe the overall purpose/motivation of the package, the structure of
the package, available functions/methods and a sample toy example.
Moreover, we have added a sample workflow using the `pbmc' dataset that
illustrates the usefulness of our package. This vignette is available
with the package and can either be viewed from the Bioconductor landing
page or from within the R console.

\hypertarget{test-package}{%
\subsubsection{Test package}\label{test-package}}

To ensure the correctness of the our package throughout its development
and post-development life-cycle, we make use of unit testing through the
R `testthat' package to test our package in small units, particularly
each function and method in our package. We particularly use
unit-testing because the github travis build allows us to automatically
run the unit tests for each commit/update and ensures that with each
update none of the functionality breaks. For each function or method
that we have either created or overrided from the other Experiment
classes, we specify a set of input and outputs and test these at each
iteration of the development life-cycle and later with the updates and
ensure that the output returned by these functions are exactly the ones
that are expected with the functionality.

The `testthat' package allows us to specify all unit tests within a
single `testthat.R' file with their corresponding inputs and the
expected outputs.

\hypertarget{results-discussion}{%
\section{Results \& Discussion}\label{results-discussion}}

\hypertarget{conclusion-limitations-future-work}{%
\section{Conclusion, limitations, \& future
work}\label{conclusion-limitations-future-work}}

\end{document}
